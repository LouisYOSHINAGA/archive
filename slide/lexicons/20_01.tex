%%% 20.1 %%%
\newcommand{\ulc}[2]{\textcolor[RGB]{#1}{\underline{\textcolor{black}{#2}}}}

\begin{frame}
    \frametitle{20.1 Defining Emotion}
    \begin{itemize}
        \item \yorangetxt{emotion}
        \begin{itemize}
            \item relatively brief episode of response to the evaluation of an external \\
                or internal event as being of major significance \ycite{Scherer, 2000}
            \item \textit{angry}, \textit{sad}, \textit{joyful}, \textit{fearful}, \textit{ashamed}, \textit{proud}, \textit{elated}, \textit{desperate}
            \yvspace{0.75}
            %
            \item emotionが判ると何が嬉しいのか:
            \begin{itemize}
                \item 言語処理タスク改善の可能性(学習支援システム,レビュー,医療)
            \end{itemize}
        \end{itemize}
    \end{itemize}
    \yvspace{0.75}
    %
    \begin{itemize}
        \item emotionの計算モデル
        \begin{enumerate}
            \item basic emotionの組み合わせ
            \item 3次元 (VAD) 空間
        \end{enumerate}
    \end{itemize}
\end{frame}


\begin{frame}
    \frametitle{20.1 Defining Emotion}
    \yhead{Basic Emotionの組み合わせ}
    %
    \begin{itemize}
        \yinner{1}
        \item emotionを 基本単位(basic emotion)の組み合わせとして表現
    \end{itemize}
    \yvspace{0.25}
    %
    \begin{itemize}
        \item 6 basic emotions \ycite{Ekman, 1999}
        \begin{itemize}
            \item surprize, happiness, anger, fear, disgust, sadness
        \end{itemize}
    \end{itemize}
    \yvspace{-0.75}
    %
    \begin{columns}
        \begin{column}[T]{0.5\textwidth}
            \yvspace{0.25}
            \begin{itemize}
                \yinner{0.4}
                \item 8 basic emotions \ycite{Plutchik, 1980}
                \begin{itemize}
                    \yinner{0.4}
                    \item \ulc{220,230,0}{joy} - \ulc{0,0,255}{sadness}
                    \item \ulc{255,0,0}{anger} - \ulc{0,128,0}{fear}
                    \item \ulc{0,255,0}{trust} - \ulc{255,0,255}{disgust}
                    \item \ulc{255,165,0}{anticipation} - \ulc{0,191,255}{surprize}
                \end{itemize}
            \end{itemize}
        \end{column}
        %
        \begin{column}[T]{0.5\textwidth}
            \yvspace{2}
            \yfig{width=0.875\textwidth}{figure/01/fig_20_2.png}
        \end{column}
    \end{columns}
\end{frame}


\begin{frame}
    \frametitle{20.1 Defining Emotion}
    \yhead{3次元 (V\,A\,D) 空間}
    \yvspace{0.25}
    %
    \begin{itemize}
        \item emotionをV, A, Dを軸とする3次元空間上の点として表現
        \begin{itemize}
            \item \underline{V}alence(感情価):刺激の心地よさ
                \yvspace{0.25}
            \item \underline{A}rousal(覚醒度):刺激による感情の強さ
                \yvspace{0.25}
            \item \underline{D}ominance(支配性):刺激に支配される度合い
            \begin{itemize}
                \item Dominanceを除いてV,Aの2次元とする場合も
            \end{itemize}
        \end{itemize}
        %
        \yvspace{-0.5}
        \begin{columns}
            \begin{column}[T]{0.575\textwidth}
                \yhspace{-1.5}
                \begin{itemize}
                    \item sentiment $\subset$ emotion
                    \begin{itemize}
                        \item valence = sentiment
                        \item emotionはsentimentの一般化
                    \end{itemize}
                \end{itemize}
            \end{column}
            %
            \begin{column}[T]{0.375\textwidth}
                \yfigcap{width=\textwidth}{figure/01/russell_circle.png}{\scriptsize Russellの円環モデル \ycite{Russell, 1980}}
            \end{column}
        \end{columns}
    \end{itemize}
\end{frame}

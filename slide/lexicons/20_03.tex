%%% 20.3 %%%

\begin{frame}
    \frametitle{20.3 Creating Affect Lexicons by Human Labeling}
    \yvspace{-0.25}
    \begin{itemize}
        \item lexicon構成手法:人手でラベリング
        \begin{itemize}
            \item crowdsourcing:小さなタスクに分割,大人数のアノテータに振り分け
        \end{itemize}
    \end{itemize}
    \yvspace{0.25}
    %
    \yhead{NRC Emotion Lexicon(EmoLex)}\ycite{Mohammad\,\&\,Turney, 2013}
    \begin{enumerate}
        \yinner{1}
        \item 同義語選択
        \begin{itemize}
            \yinner{0.5}
            \item アノテータのword sense検証,問題は自動生成
        \end{itemize}
        \yvspace{-0.25}
        %
        \begin{description}
            \yinner{-3.2}
            \footnotesize
            \item[例] \texttt{Which word is closest in meaning (most related) to startle?} \\
                \texttt{automobile} / \texttt{shake} / \texttt{honesty} / \texttt{entertain}
        \end{description}
        %
        \item 8 basic emotionsとの関連度の回答
        \begin{itemize}
            \item 関連度:\textit{not}, \textit{weakly}, \textit{moderately}, \textit{strongly}
        \end{itemize}
    \end{enumerate}
    \yvspace{-0.25}
    %
    \begin{columns}
        \begin{column}[T]{0.7\textwidth}
            \begin{itemize}
                \yinner{1.25}
                % \item 集計,外れ値を除外
                \item 多数決で関連度を決定
                \item 関連度よりスコア$\in \{0, 1\}$を決定
                \begin{itemize}
                    \yinner{1}
                    \item \textit{not}, \textit{weakly} $\mapsto 0$
                    \item \textit{moderately}, \textit{strongly} $\mapsto 1$
                \end{itemize}
            \end{itemize}
        \end{column}
        %
        \begin{column}[T]{0.3\textwidth}
            \yvspace{-1}
            \yfig{width=\textwidth}{figure/02/emolex.png}
        \end{column}
    \end{columns}
\end{frame}


\begin{frame}
    \frametitle{20.3 Creating Affect Lexicons by Human Labeling}
    \yvspace{-0.25}
    \begin{itemize}
        \item lexicon構成手法:人手でラベリング
        \begin{itemize}
            \item crowdsourcing:小さなタスクに分割,大人数のアノテータに振り分け
        \end{itemize}
    \end{itemize}
    \yvspace{0.25}
    %
    \yhead{NRC VAD Lexicon} \ycite{Mohammad, 2018}
    \begin{enumerate}
        \yinner{1.25}
        \item $N$個{\footnotesize(通常$N=4$)}のwordの中からbest/worstを選択
        \begin{description}
            \yinner{0.25}
            \footnotesize
            \item[例:Valence] \
            \begin{itemize}
                \yinner{-5.75}
                \scriptsize
                \item Which of the four words below is associated with the MOST happiness / pleasure / positiveness / satisfaction / contentedness / hopefulness OR LEAST unhappiness / annoyance / negativeness / dissatisfaction / melancholy / despair ?
                    \yvspace{0.25}
                \item Which of the four words below is associated with the LEAST happiness / pleasure / positiveness / satisfaction / contentedness / hopefulness OR MOST unhappiness / annoyance / negativeness / dissatisfaction / melancholy / despair ?
            \end{itemize}
        \end{description}
    \end{enumerate}
    \yvspace{-0.25}
    %
    \begin{itemize}
        \yinner{1}
        \item best-worst scaling
        \begin{itemize}
            \yinner{1}
            \item $(\textrm{wordのスコア}) = (\textrm{bestに選ばれた割合}) - (\textrm{worstに選ばれた割合})$
        \end{itemize}
    \end{itemize}
    \yvspace{-0.75}
    %
    \begin{columns}
        \begin{column}[T]{0.5\textwidth}
            \begin{itemize}
                \yinner{1.4}
                \item 評価:split-half reliability
                \begin{itemize}
                    \yinner{1}
                    \item コーパスを2分割 \\
                        相関を計算
                \end{itemize}
            \end{itemize}
        \end{column}
        %
        \begin{column}[T]{0.5\textwidth}
            \yvspace{-0.5}
            \yfig{width=0.95\textwidth}{figure/02/fig_20_4.png}
        \end{column}
    \end{columns}
\end{frame}

%%% 20.4 %%%

\begin{frame}
    \frametitle{20.4 Semi-supervised Induction of Affect Lexicons}
    \yvspace{0.25}
    \yhead{20.4.1 Semantic Axis Method}
    \begin{enumerate}
        \yinner{1}
        \item seed wordsを人手で選択
        \begin{description}
            \yinner{-2.25}
            \item[(1)] single large seed lexiconを使用
            \item[(2)] 分野ごとにseed wordsを使い分ける
        \end{description}
        \yvspace{-0.25}
        \yfigcap{width=0.7\textwidth}{figure/04/seed_words.png}{分野によるseed words例 \ycite{Hamilton+, 2016}}
        \yvspace{0.75}
        %
        \item 埋め込み空間での計算
        \begin{itemize}
            \item seed wordsを埋め込み,重心$\bm{V}^{+}, \bm{V}^{-}$,semantic axis $\bm{V}_{axis}$を計算
            \yvspace{-0.25}
            {\footnotesize
                \begin{align*}
                    \bm{V}^{+} = \frac{1}{n} \sum_{1}^{n} E(w_{i}^{+}), \quad
                    \bm{V}^{-} = \frac{1}{m} \sum_{1}^{m} E(w_{i}^{-}), \quad
                    \bm{V}_{axis} = \bm{V}^{+} - \bm{V}^{-}
                \end{align*}
            }
        \end{itemize}
    \end{enumerate}
\end{frame}


\begin{frame}
    \frametitle{20.4 Semi-supervised Induction of Affect Lexicons}
    \begin{itemize}
        \item seed wordsを埋め込み,重心$\bm{V}^{+}, \bm{V}^{-}$,semantic axis $\bm{V}_{axis}$を計算
        \yvspace{-0.25}
        {\footnotesize
            \begin{align*}
                \bm{V}^{+} = \frac{1}{n} \sum_{1}^{n} E(w_{i}^{+}), \quad
                \bm{V}^{-} = \frac{1}{m} \sum_{1}^{m} E(w_{i}^{-}), \quad
                \bm{V}_{axis} = \bm{V}^{+} - \bm{V}^{-}
            \end{align*}
        }
        \item word $w$のスコアを計算
        {\footnotesize
            \begin{yalign*}
                \mathrm{score}(w) = \cos(E(w), \bm{V}_{axis}) = \frac{E(w) \cdot \bm{V}_{axis}}{\|E(w)\| \|\bm{V}_{axis}\|}
            \end{yalign*}
        }
        \yvspace{-0.25}
        \yfig{width=0.875\textwidth}{figure/04/axis.png}
    \end{itemize}
\end{frame}


\begin{frame}
    \frametitle{20.4 Semi-supervised Induction of Affect Lexicons}
    \yhead{20.4.2 Label Propagation}
    \begin{enumerate}
        \yinner{1}
        \item グラフ定義
        \begin{itemize}
            \item node:word embedding $\bm{w}$,$k$個の近傍nodeと接続
            \item edge:node $\bm{w}_i$, $\bm{w}_j$間の重み$E_{i,j}$
            \begin{yalign*}
                E_{i,j} = \arccos\left(- \frac{\bm{w}_{i}^{\top} \bm{w}_{j}}{\|\bm{w}_{i}\| \|\bm{w}_{j}\|} \right)
            \end{yalign*}
        \end{itemize}
        %
        \item seed set定義
        \item seed setからpolarityを伝播  \label{label_prop_propagate}
        \begin{itemize}
            \item seed setを起点に edgeの重みに従ってrandom walk
        \end{itemize}
        \yfig{width=0.5\textwidth}{figure/04/fig_20_7a.png}
    \end{enumerate}
\end{frame}


\begin{frame}
    \frametitle{20.4 Semi-supervised Induction of Affect Lexicons}
    \begin{enumerate}
        \setcounter{enumi}{3}
        \item word scoreの計算  \label{label_prop_word_score}
        \begin{itemize}
            \item random walkの訪問回数からスコアを計算
            \begin{yalign*}
                \mathrm{score}^{+}(w_{i}) = \frac{\mathrm{rawscore}^{+}(w_i)}{\mathrm{rawscore}^{+}(w_i) + \mathrm{rawscore}^{-}(w_i)}
            \end{yalign*}
            \yvspace{-0.25}
            \yfig{width=0.5\textwidth}{figure/04/fig_20_7b.png}
        \end{itemize}
        \yvspace{0.5}
        %
        \item スコアの信頼度の計算
        \begin{itemize}
            \item word scoreはseed set依存
            \item seed setの部分集合をrandomに選びpropagation (\ybluetxt{\ref{label_prop_propagate}}-\ybluetxt{\ref{label_prop_word_score}}) を複数回実行
            \item 各wordのスコアの標準偏差を計算
        \end{itemize}
    \end{enumerate}
\end{frame}


\begin{frame}
    \frametitle{20.4 Semi-supervised Induction of Affect Lexicons}
    \yhead{20.4.3 Other Methods}
    \begin{itemize}
        \yinner{1}
        \item seed setとの\underline{類似度}の尺度
    \end{itemize}
    \yvspace{0.25}
    %
    \begin{itemize}
        \yinner{1}
        \item and, butによる形容詞の並列関係 \ycite{Hatzivassiloglou\,\&\,McKeown, 1997}
        \begin{itemize}
            \yinner{1}
            \item and:同じpolarity
            \begin{itemize}
                \yinner{1}
                \item fair and legitimate / corrupt and brutal
            \end{itemize}
            \item but:逆のpolarity
            \begin{itemize}
                \yinner{1}
                \item fair but brutal
            \end{itemize}
        \end{itemize}
    \end{itemize}
    \yvspace{-0.25}
    %
    \begin{itemize}
        \yinner{1}
        \item 同一の語幹の形容詞のmorphological negation
        \begin{itemize}
            \yinner{1}
            \item \textit{un-}, \textit{im-}, \textit{-less}がつくと逆のpolarity
            \begin{itemize}
                \yinner{1}
                \item adequate/inadequate, \ thoughtful/thoughtless
            \end{itemize}
        \end{itemize}
    \end{itemize}
    \yvspace{-0.25}
    %
    \begin{itemize}
        \yinner{1}
        \item WordNetのsynonim/antonym
        \begin{itemize}
            \yinner{1}
            \item synonim:同じpolarity
            \item antonym:逆のpolarity
        \end{itemize}
        \quad $\to$ SentiWordNet \ycite{Baccianella+, 2010}
    \end{itemize}
\end{frame}

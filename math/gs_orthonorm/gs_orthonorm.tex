%%% math support content: Gram-Schmidt orthonormalization %%% 
\documentclass[dvipdfmx]{jsarticle}
\usepackage[top=20truemm, bottom=20truemm, left=20truemm, right=20truemm]{geometry}
\usepackage{amsmath, amssymb, mathtools, bm}
\usepackage{ascmac}
\usepackage{framed}
\usepackage[dvipdfmx]{graphicx, xcolor}
\usepackage{here}
\usepackage{algorithm, algorithmic}
\usepackage{url}
\usepackage[dvipdfmx]{hyperref}
\usepackage{pxjahyper}
\usepackage{comment}

\usepackage{amsthm}
\theoremstyle{definition}
\newtheorem{definition}{定義}[section]
\newtheorem*{definition*}{定義}
\newtheorem{theorem}[definition]{定理}
\newtheorem*{theorem*}{定理}
\newtheorem{proposition}[definition]{命題}
\newtheorem*{proposition*}{命題}
\newtheorem{example}[definition]{例}
\newtheorem*{example*}{例}
\newtheorem{question}{問}[section]

\usepackage{tikz}
\usepackage{tikz-3dplot}
\usetikzlibrary{intersections, calc, arrows}


\title{完全理解 正規直交化}
\author{吉永 塁 \thanks{情報工学課程4年 \texttt{b7122062@edu.kit.ac.jp}}}
\date{\today}


\begin{document}

\maketitle{}

%%%======================================================================
\section{はじめに}
%%%======================================================================
シュミットの正規直交化(教科書\cite{la} pp.116-117)と,それに関連する項目を中心にまとめました.
できるだけこの中で完結するように,必要な部分は引用して示すようにしました.
記号の表記などは教科書に合わせました.
副読本的な使い方をしていただければ良いかと思います.
理解の一助になれば幸いです.



%%%======================================================================
\section{関連項目}
%%%======================================================================

%%%----------------------------------------------------------------------
\subsection{内積空間}
%%%----------------------------------------------------------------------
教科書の内積($\bm{R}$上のベクトル空間の内積)の定義を以下に示します.

\begin{leftbar}
    \begin{definition*}[内積]
        $\bm{R}$上のベクトル空間$V$の2つのベクトル$\bm{u}, \bm{v}$に対して実数$(\bm{u}, \bm{v})$を対応させる対応$( \quad , \quad )$が次の4条件を満たすとき,
        ベクトル空間$V$の内積という.
        \begin{align*}
            (1) \quad &(\bm{u} + \bm{u}', \bm{v}) = (\bm{u}, \bm{v}) + (\bm{u}', \bm{v}) \\
            (2) \quad &(c \bm{u}, \bm{v}) = c (\bm{u}, \bm{v}) \\
            (3) \quad &(\bm{v}, \bm{u}) = (\bm{u}, \bm{v}) \\
            (4) \quad &\bm{u} \neq \bm{0} \text{ならば} (\bm{u}, \bm{u}) > 0
        \end{align*}
        ここで$\bm{u}, \bm{u}', \bm{v} \in V, \ c \in \bm{R}$である.特に(2),(3)より$(\bm{u}, \bm{0}) = (\bm{0}, \bm{u}) = \bm{0}$がわかる.
        内積をもつベクトル空間を内積空間という.
    \end{definition*}
\end{leftbar}

上の条件について,(1)(2)を線形性,(3)を対称性,(4)を正定値性と呼ぶことがあります.

また,$\bm{C}$上のベクトル空間における内積(エルミート内積)を考える場合には,上に示した$\bm{R}$上の内積空間の定義について,
\begin{itemize}
    \item 「$\bm{R}$」を「$\bm{C}$」
    \item 「実数」を「複素数」
    \item (3)の式を $(\bm{v}, \bm{u}) = \overline{(\bm{u}, \bm{v})}$
\end{itemize}
へと置き換えてください.
また,このときの(3)を歪対称性と呼ぶことがあります.

シュミットの正規直交化は,内積空間での話であることに注意してください.
教科書では$\bm{R}$上の内積空間で,内積を$\bm{R}^n$の標準内積として書かれていますが,
正規直交化において,内積は$\bm{R}^n$の標準内積である必要性はなく,上の条件
\footnote{
    考えているベクトル空間が$\bm{R}$上のものか$\bm{C}$上のものかによって内積が満たすべき条件(特に(3))が変わることに注意してください.
}
を満たす任意の内積で成立します
\footnote{
    シュミットの正規直交化の証明の過程で,内積がどのようなものであるのかについては言及されていないことに注意してください.
}.
例えば,$\bm{C}$上の内積空間で,$\bm{u}, \bm{v} \in \bm{C}^n$に対し,
\begin{align}
    (\bm{u}, \bm{v}) \coloneqq {}^{t}\bm{u} \overline{\bm{v}} = \sum_{i=1}^n u_i \overline{v_i}
    \label{norm_hermit}
\end{align}
で定められる$\bm{C}^n$の標準エルミート内積を用いた場合や,
$n$次以下の実数係数多項式空間$\bm{R}[x]_n$で,$f, g \in \bm{R}[x]_n$に対し,
\begin{align}
    (f, g) \coloneqq \int_{-1}^{1} f(x) g(x) dx
    \label{func_ip}
\end{align}
で定められる内積(教科書p.112 例題6.1.2)を用いる場合に対しても適用できるということです.


%%%----------------------------------------------------------------------
\subsection{正規直交基}
%%%----------------------------------------------------------------------
教科書の正規直交基の定義を以下に示します.

\begin{leftbar}
    \begin{definition*}[正規直交基]
        次の条件($\ast$)を満たす$V$の基$\{ \bm{u}_1, \cdots, \bm{u}_n \}$を正規直交基という.
        \begin{align}
            (\bm{u}_i, \bm{u}_j) = \delta_{ij} \quad (1 \leq i, \, j \leq n) \tag{$\ast$}
        \end{align}
    \end{definition*}
\end{leftbar}

ここで$\delta_{ij}$は,次のように定義されるクロネッカーのデルタです.
\begin{align*}
    \delta_{ij} =
    \begin{cases}
        1 & (i = j) \\
        0 & (i \neq j)
    \end{cases}
\end{align*}

つまり,基$\{ \bm{u}_1, \ldots, \bm{u}_n \}$で,同じものとの内積$(\bm{u}_i, \bm{u}_i)$は1となり,
異なるものとの内積$(\bm{u}_i, \bm{u}_j) \ (i \neq j)$が0となるようなものを正規直交基と呼ぶということです.
$(\bm{u}_i, \bm{u}_i) = 1$より,$\| \bm{u}_i \| = 1$なので,正規化
\footnote{(方向は変えずに)ベクトルのノルムが1になるように変換(=単位ベクトル化)する操作のことを正規化と呼びます.}
された状態であって,$(\bm{u}_i, \bm{u}_j) = 0 \ (i \neq j)$なので,自身以外のベクトルと直交
\footnote{直交の定義は,$\bm{u} \perp \bm{v} :\Longleftrightarrow (\bm{u}, \bm{v}) = 0$.これも内積の選び方には依存しないことに注意.}
するという性質を持つことが正規直交基と呼ばれる所以です.

教科書に載っていない正規直交基の具体例をいくつか挙げておきます.

\begin{example}
    次の基は$\bm{C}^n$の正規直交基である.
    但し,$\omega = \exp{\dfrac{2\pi i}{n}} = \cos{\dfrac{2\pi}{n}} + i \sin{\dfrac{2\pi}{n}}$で,内積は(\ref{norm_hermit})式.
    \begin{align*}
        \left\{
            \frac{1}{\sqrt{n}} \begin{bmatrix} 1 \\ 1 \\ \vdots \\ 1 \end{bmatrix} ,
            \frac{1}{\sqrt{n}} \begin{bmatrix} 1 \\ \omega \\ \vdots \\ \omega^{n-1} \end{bmatrix} ,
            \frac{1}{\sqrt{n}} \begin{bmatrix} 1 \\ \omega^2 \\ \vdots \\ \omega^{2(n-1)} \end{bmatrix} ,
            \cdots ,
            \frac{1}{\sqrt{n}} \begin{bmatrix} 1 \\ \omega^{n-1} \\ \vdots \\ \omega^{(n-1)(n-1)} \end{bmatrix}
        \right\}
    \end{align*}
\end{example}

\vspace{\baselineskip}

\begin{example}
    \label{ex_polspaceno}
    基$\{ E_0, E_1, \ldots, E_n \}$は,多項式空間$\bm{R}[x]_n$の正規直交基である.
    但し,$E_k$は次の式で定められるもので,内積は(\ref{func_ip})式.
    \begin{align*}
        E_k(x) = \sqrt{\frac{2k+1}{2}} \frac{1}{2^k k!} \frac{d^k}{dx^k} (x^2 - 1)^k
    \end{align*}
\end{example}

\vspace{\baselineskip}

\begin{example}
    $a_n, b_n \in \bm{R}$で,
    $f(x) = \dfrac{1}{2} a_0 + {\displaystyle \sum_{n=1}^N} (a_n \cos(nx) + b_n \sin(nx))$
    なる形の関数全体が作るベクトル空間$V_N$に,内積を
    $(f, g) \coloneqq {\displaystyle \int_{-\pi}^{\pi}} f(x) g(x) dx$
    で定めるとき,次の基は$V_N$の正規直交基である.
    \begin{align*}
        \left\{
            \frac{1}{\sqrt{\pi}},
            \frac{1}{\sqrt{\pi}} \cos(x),
            \frac{1}{\sqrt{\pi}} \sin(x),
            \ldots,
            \frac{1}{\sqrt{\pi}} \cos(nx),
            \frac{1}{\sqrt{\pi}} \sin(nx)
        \right\}
    \end{align*}
\end{example}


%%%======================================================================
\section{シュミットの正規直交化}
%%%======================================================================

%%%----------------------------------------------------------------------
\subsection{定理}
%%%----------------------------------------------------------------------
教科書の定理6.2.1(シュミットの正規直交化)を以下に示します.

\begin{itembox}[l]{定理6.2.1 \ シュミットの正規直交化}
$V$の一組の基を$\{\bm{v}_1, \ldots, \bm{v}_n\}$とすると,正規直交基$\{\bm{u}_1, \ldots, \bm{u}_n\}$で
\begin{align*}
    \langle \bm{v}_1, \ldots, \bm{v}_n \rangle_{\bm{R}} = \langle \bm{u}_1, \ldots, \bm{u}_n \rangle_{\bm{R}} \quad (1 \leq r \leq n)
\end{align*}
となるものが存在する.
特に有限次元の内積空間は正規直交基を持つ.
\end{itembox}

この定理が主張しているのは.
「(正規直交基ではない)基$\{\bm{v}_1, \ldots, \bm{v}_n\}$で生成される部分空間$V$は,ある正規直交基$\{\bm{u}_1, \ldots, \bm{u}_n\}$でも生成される」
ということです.
別の見方をすれば,
「部分空間$V$を生成する基$\{\bm{v}_1, \ldots, \bm{v}_n\}$を,適当な正規直交基$\{\bm{u}_1, \ldots, \bm{u}_n\}$に変換することができる」
ということです.
もう少し具体的に言うと,
「$V$の基を構成するベクトル$\bm{v}_i \, (i = 1, \ldots, n)$は,適当な正規直交基$\{\bm{u}_1, \ldots, \bm{u}_n\}$によって表される」
ということになります.
そしてこの,$\bm{v}_i$を$\bm{u}_1, \ldots, \bm{u}_n$によって表していく操作がシュミットの正規直交化であるということです.


%%%----------------------------------------------------------------------
\subsection{証明}
%%%----------------------------------------------------------------------
定理の証明が正規直交化の手順そのものとなっています.
教科書を順に追っていきます.
\begin{leftbar}
    まず
    \begin{align*}
        \bm{u}_1 = \bm{v}_1 / \|\bm{v}_1\|
    \end{align*}
    とおくと,$\|\bm{u}_1\| = 1$である.
\end{leftbar}
この操作に関しては問題ないと思います.
$\bm{v}_1$を正規化しているだけです.

\begin{leftbar}
    次に
    \begin{align}
        \bm{v}_2' = \bm{v}_2 - (\bm{v}_2, \bm{u}_1)\bm{u}_1,\quad \bm{u}_2 = \bm{v}_2' / \|\bm{v}_2'\|
        \label{proof_delelem}
    \end{align}
    とおくと,$(\bm{u}_1, \bm{u}_2) = 0$,$\|\bm{u}_2\| = 1$であり
\end{leftbar}

$\| \bm{u}_1 \| = 1$に注意して,(\ref{proof_delelem})式左を眺めてみると,右辺第二項が$\bm{v}_2$の$\bm{u}_1$への正射影となっていることがわかります.
つまり,$(\bm{v}_2, \bm{u}_1) \bm{u}_1$の項は,$\bm{v}_2$の中に含まれている$\bm{u}_1$方向のベクトルを表しています.
$(\bm{v}_2, \bm{u}_1)$が大きさ,$\bm{u}_1$が方向をそれぞれ表しています.
よって,(\ref{proof_delelem})式左は,$\bm{v}_2$から$\bm{u}_1$の成分を削りとるという操作(直交化)を表しています.
(\ref{proof_delelem})式右では,そのようにして得られた$\bm{v}_2'$を正規化しています.

この操作によって,$\bm{u}_1$の成分を含まない$\bm{u}_2$が得られるので,$(\bm{u}_1, \bm{u}_2) = 0$となります.
実際に,
\begin{align*}
    (\bm{u}_1, \bm{u}_2)
    = \left( \bm{u}_1, \frac{\bm{v}_2'}{\|\bm{v}_2'\|} \right)
    = \frac{1}{\|\bm{v}_2'\|} (\bm{u}_1, \bm{v}_2 - (\bm{v}_2, \bm{u}_1) \bm{u}_1)
    = \frac{1}{\|\bm{v}_2'\|} ((\bm{u}_1, \bm{v}_2) - (\bm{v}_2, \bm{u}_1)\|\bm{u}_1\|^2)
    = 0
\end{align*}

からも$\bm{u}_1$と$\bm{u}_2$が直交していることがわかります.

\begin{leftbar}
    さらに
    \begin{align}
        \langle \bm{u}_1, \bm{u}_2 \rangle_{\bm{R}}
        = \langle \bm{u}_1, \bm{v}_2 \rangle_{\bm{R}}
        = \langle \bm{v}_1, \bm{v}_2 \rangle_{\bm{R}} .
        \label{proof_vseq}
    \end{align}
\end{leftbar}

これが本当に成り立っているか,具体的に見てみましょう.
まずは$\bm{u}_2$の作り方を思い出してみます.
$\bm{u}_2$は$\bm{v}_2'$から作られ,$\bm{v}_2'$は$\bm{u}_1, \bm{v}_2$より作られたのでした.
つまり,$\bm{u}_2$は$\bm{u}_1, \bm{v}_2$によって表されるので,
このことから(\ref{proof_vseq})式左側の$\langle \bm{u}_1, \bm{u}_2 \rangle_{\bm{R}} = \langle \bm{u}_1, \bm{v}_2 \rangle_{\bm{R}}$がわかります.
次に,$\bm{u}_1$の作り方についてですが,これは$\bm{v}_1$の正規化によって得られるので,
(\ref{proof_vseq})式右側の$\langle \bm{u}_1, \bm{v}_2 \rangle_{\bm{R}} = \langle \bm{v}_1, \bm{v}_2 \rangle_{\bm{R}}$もわかります.

何故ベクトルの移り変わりだけを見れば良いのかということについて,(\ref{proof_vseq})式右側
\footnote{簡単なので右側を選びましたが左側でも同じことです.}
を例にして考えてみましょう.
$\bm{u}_1, \bm{v}_2$が生成する空間を$W$とします.
$\bm{u}_1, \bm{v}_2$が空間$W$を生成するとは,$W$に属する任意のベクトルが$\bm{u}_1, \bm{v}_2$の一次結合で表されるということです.
式を使って書くと,任意の$\bm{w} \in W$が,$\bm{w} = c_1 \bm{u}_1 + c_2 \bm{v}_2$のような形で表されるということです($c_1, c_2$は係数).
いま確かめたいのは,$\bm{v}_1, \bm{v}_2$が$W$を生成するのかということなので,
任意の$\bm{w} \in W$について,適当な係数$d_1, d_2$を用いて,
\begin{align*}
    \bm{w} = c_1 \bm{u}_1 + c_2 \bm{v}_2 = d_1 \bm{v}_1 + d_2 \bm{v}_2
\end{align*}
と書き直すことができるかを考えれば良いということになります.
よって,ベクトルの移り変わりのみを考えればよく,この例では$\bm{u}_1 = \bm{v}_1 / \| \bm{v}_1 \|$なので,
\begin{align*}
    c_1 \bm{u}_1 + c_2 \bm{v}_2 = c_1 \frac{\bm{v}_1}{\| \bm{v}_1 \|} + c_2 \bm{v}_2
\end{align*}
となります($d_1 = c_1 / \| \bm{v}_1 \|, \ d_2 = c_2$).


\begin{leftbar}
    これを続けて,$\bm{u}_1, \ldots, \bm{u}_r \ (1 \leq r < n)$が求まったとき,
    \begin{align*}
        \bm{v}_{r+1}' &= \bm{v}_{r+1} - \sum_{i=1}^r (\bm{v}_{r+1}, \bm{u}_i) \bm{u}_i \\
        \bm{u}_{r+1} &= \bm{v}_{r+1}' / \|\bm{v}_{r+1}'\|
    \end{align*}
    とおく,$(\bm{v}_{r+1}', \bm{u}_i) = 0 \ (1 \leq i \leq r)$だから$(\bm{u}_{r+1}, \bm{u}_i) = 0$であり,
    \begin{align*}
        \langle \bm{u}_1, \cdots, \bm{u}_r, \bm{u}_{r+1} \rangle_{\bm{R}}
        &= \langle \bm{u}_1, \cdots, \bm{u}_r, \bm{v}_{r+1} \rangle_{\bm{R}} \\
        &= \langle \bm{v}_1, \cdots, \bm{v}_{r+1} \rangle_{\bm{R}}
    \end{align*}
    が成り立っている.
    これを続けて正規直交基を得る.
    \hfill $\blacksquare$
\end{leftbar}

$\bm{u}_3, \ldots, \bm{u}_n$に関しても,同様の操作を繰り返して求めれば良いということです
($r = 1$としたのが(\ref{proof_delelem}),(\ref{proof_vseq})式です).

%%%----------------------------------------------------------------------
\subsection{まとめ}
\label{ssec_conc}
%%%----------------------------------------------------------------------
定理6.2.1の証明から,正規直交化の計算に関係する部分のみを示します
\footnote{教科書では,$\bm{u}_{r+1}$の求め方として書かれていますが,添字をずらして$\bm{u}_{j}$の求め方として書きました.}.
\begin{itembox}[l]{シュミットの正規直交化(計算手順)}
    \begin{enumerate}
        \item $\bm{u}_1 = \dfrac{\bm{v}_1}{\| \bm{v}_1 \|}$を計算する.  \label{calc_u1}
        \item $j = 2, \ldots, n$について,次の手順\ref{calc_vjd},\ref{calc_uj}を繰り返し行う.
        \item $\bm{v}_j' = \bm{v}_j - {\displaystyle \sum_{k=1}^{j-1}} (\bm{v}_j, \bm{u}_k) \bm{u}_k$を計算する.  \label{calc_vjd}
        \item $\bm{u}_j = \dfrac{\bm{v}_j'}{\| \bm{v}_j' \|}$を計算する.  \label{calc_uj}
    \end{enumerate}
\end{itembox}
証明では正規直交基を求める計算以外の記述も混ざっていたので複雑に思えたかもしれませんが,正規直交化の操作自体はとても単純なものです.

シュミットの正規直交化の手順をアルゴリズムとして示しておきます.
\footnote{
    教科書の証明では$\bm{u}_1$と$\bm{u}_j (j=2,\ldots,n)$を求める手順は別個に書かれていますが,
    計算自体は同じ形なので,場合分けを使うことでまとめて記述しました.
}

\begin{algorithm}
    \caption{シュミットの正規直交化}
    \label{schmidt-orthonormalization}
    \begin{algorithmic}
        \renewcommand{\algorithmicrequire}{\textbf{Input:}}
        \renewcommand{\algorithmicensure}{\textbf{Output:}}
        \REQUIRE 基$\{ \bm{v}_1, \ldots, \bm{v}_n \}$
        \ENSURE 正規直交基$\{ \bm{u}_1, \ldots, \bm{u}_n \}$
        \FOR{$j = 1$ to $n$}
            \STATE $\bm{v}_j' \leftarrow \bm{v}_j$
            \IF{$j \neq 1$}
                \FOR{$k = 1$ to $j-1$}
                    \STATE $\bm{v}_j' \leftarrow \bm{v}_j' - (\bm{v}_j, \bm{u}_k) \bm{u}_k$
                \ENDFOR
            \ENDIF
            \STATE $\bm{u}_j \leftarrow \bm{v}_j' / \| \bm{v}_j' \|$
        \ENDFOR
    \end{algorithmic}
\end{algorithm}


\begin{question}
    アルゴリズム\ref{schmidt-orthonormalization}を参考に,シュミットの正規直交化を行うプログラムを作成してください.
\end{question}



%%%----------------------------------------------------------------------
\subsection{計算と可視化}
%%%----------------------------------------------------------------------
教科書の例題をもとに,具体的な計算方法を見ていきます.
同時に,ベクトルの移り変わりを図に描いて確認していきます.

\begin{itembox}[l]{例題6.2.1}
    シュミットの正規直交化を用いて,$\bm{R}^3$の次の基を正規直交化せよ.
    \begin{align*}
        \left\{
            \begin{bmatrix} 1 \\ 1 \\ 0 \end{bmatrix} ,
            \begin{bmatrix} 1 \\ 3 \\ 1 \end{bmatrix} ,
            \begin{bmatrix} 2 \\ -1 \\ 1 \end{bmatrix}
        \right\}
    \end{align*}
\end{itembox}


\begin{figure}[H]
    % left
    \begin{minipage}{0.60\textwidth}
        \begin{leftbar}
            \begin{align*}
                \bm{v}_1 = \begin{bmatrix} 1 \\ 1 \\ 0 \end{bmatrix} ,
                \bm{v}_2 = \begin{bmatrix} 1 \\ 3 \\ 1 \end{bmatrix} ,
                \bm{v}_3 = \begin{bmatrix} 2 \\ -1 \\ 1 \end{bmatrix}
            \end{align*}
            とおく.定理6.2.1のように$\bm{u}_1, \bm{u}_2, \bm{u}_3$を順に求めていく.
        \end{leftbar}

        最初の状態($\bm{v}_1, \bm{v}_2, \bm{v}_3$)を図で表すと図\ref{fig_v1v2v3}のようになります.
        $\{ \bm{v}_1, \bm{v}_2, \bm{v}_3 \}$は$\bm{R}^3$の基にはなっていますが,ベクトルの大きさもばらばらで直交もしていません.
    \end{minipage}
    % right
    \begin{minipage}{0.39\textwidth}
        \begin{center}
            \tdplotsetmaincoords{70}{100}
            \begin{tikzpicture}[tdplot_main_coords, scale=0.85]
                % axis
                \draw[thick, ->] (-2,0,0) -- (4,0,0) node[anchor=north east]{$x$};  % x-axis
                \draw[thick, ->] (0,-2,0) -- (0,3.5,0) node[anchor=north west]{$y$};  % y-axis
                \draw[thick, ->] (0,0,-1) -- (0,0,2) node[anchor=south]{$z$};  % z-axis
                % origin
                \node at(-0.5,-0.3,0.05) {{\footnotesize $O$}};
                % vector v_1
                \draw[very thin, dashed, color=gray] (1,0,0) -- (1,1,0) -- (0,1,0);
                \draw[very thick, ->, color=red] (0,0,0) -- (1,1,0);
                \node at(1.5,1.2,0) {$\bm{v}_1$};
                % vector v_2
                \draw[very thin, dashed, color=gray] (1,0,0) -- (1,3,0) -- (0,3,0);
                \draw[very thin, dashed, color=gray] (1,3,0) -- (1,3,1);
                \draw[very thick, ->, color=green] (0,0,0) -- (1,3,1);
                \node at(1.2,3.2,1.2) {$\bm{v}_2$};
                % vector v_3
                \draw[very thin, dashed, color=gray] (2,0,0) -- (2,-1,0) -- (0,-1,0);
                \draw[very thin, dashed, color=gray] (2,-1,0) -- (2,-1,1);
                \draw[very thick, ->, color=blue] (0,0,0) -- (2,-1,1);
                \node at(2.2,-1.2,1.2) {$\bm{v}_3$};
            \end{tikzpicture}
        \end{center}
        \caption{$\bm{v}_1, \bm{v}_2, \bm{v}_3$}
        \label{fig_v1v2v3}
    \end{minipage}
\end{figure}

\begin{figure}[H]
    % left
    \begin{minipage}{0.60\textwidth}
        \begin{leftbar}
            \begin{align*}
                \bm{u}_1
                = \frac{1}{\| \bm{v}_1 \|} \begin{bmatrix} 1 \\ 1 \\ 0 \end{bmatrix}
                = \frac{1}{\sqrt{2}} \begin{bmatrix} 1 \\ 1 \\ 0 \end{bmatrix}
            \end{align*}
        \end{leftbar}

        最初に$\bm{v}_1$の正規化を行い$\bm{u}_1$を求めます.
        ノルムの計算は定義より,$\| \bm{v}_1 \| = \sqrt{(\bm{v}_1, \bm{v}_1)}$で計算されます.
        この操作は,\ref{ssec_conc}節の手順\ref{calc_u1}に当たります.
    \end{minipage}
    % right
    \begin{minipage}{0.39\textwidth}
        \begin{center}
            \tdplotsetmaincoords{70}{100}
            \begin{tikzpicture}[tdplot_main_coords, scale=0.85]
                % axis
                \draw[thick, ->] (-2,0,0) -- (4,0,0) node[anchor=north east]{$x$};  % x-axis
                \draw[thick, ->] (0,-2,0) -- (0,3.5,0) node[anchor=north west]{$y$};  % y-axis
                \draw[thick, ->] (0,0,-1) -- (0,0,2) node[anchor=south]{$z$};  % z-axis
                % origin
                \node at(-0.5,-0.3,0.05) {{\footnotesize $O$}};
                % vector u_1
                \draw[very thin, dashed, color=gray] (0.7071,0,0) -- (0.7071,0.7071,0) -- (0,0.7071,0);
                \draw[very thick, ->, color=orange] (0,0,0) -- (0.7071,0.7071,0);
                \node at(0.8,1,0) {$\bm{u}_1$};
                % vector v_2
                \draw[very thin, dashed, color=gray] (1,0,0) -- (1,3,0) -- (0,3,0);
                \draw[very thin, dashed, color=gray] (1,3,0) -- (1,3,1);
                \draw[very thick, ->, color=green] (0,0,0) -- (1,3,1);
                \node at(1.2,3.2,1.2) {$\bm{v}_2$};
                % vector v_3
                \draw[very thin, dashed, color=gray] (2,0,0) -- (2,-1,0) -- (0,-1,0);
                \draw[very thin, dashed, color=gray] (2,-1,0) -- (2,-1,1);
                \draw[very thick, ->, color=blue] (0,0,0) -- (2,-1,1);
                \node at(2.2,-1.2,1.2) {$\bm{v}_3$};
            \end{tikzpicture}
        \end{center}
        \caption{$\bm{u}_1, \bm{v}_2, \bm{v}_3$}
        \label{fig_u1v2v3}
    \end{minipage}
\end{figure}

\begin{figure}[H]
    % left
    \begin{minipage}{0.60\textwidth}
        \begin{leftbar}
            \begin{align*}
                \bm{v}_2'
                = \bm{v}_2 - (\bm{v}_2, \bm{u}_1) \bm{u}_1
                = \begin{bmatrix} 1 \\ 3 \\ 1 \end{bmatrix}
                    - \frac{4}{\sqrt{2}} \frac{1}{\sqrt{2}} \begin{bmatrix} 1 \\ 1 \\ 0 \end{bmatrix}
                = \begin{bmatrix} -1 \\ 1 \\ 1 \end{bmatrix} ,
            \end{align*}
        \end{leftbar}

        次に,$\bm{v}_2$から$\bm{u}_1$の成分を引くことで$\bm{v}_2'$を求めます.
        図だけでは読み取りにくいかも知れませんが,$\bm{u}_1$と$\bm{v}_2'$が直交するようになっています.
        この操作は,\ref{ssec_conc}節の手順\ref{calc_vjd}で$j = 2$の場合に相当します.
    \end{minipage}
    % right
    \begin{minipage}{0.39\textwidth}
        \begin{center}
            \tdplotsetmaincoords{70}{100}
            \begin{tikzpicture}[tdplot_main_coords, scale=0.85]
                % axis
                \draw[thick, ->] (-2,0,0) -- (4,0,0) node[anchor=north east]{$x$};  % x-axis
                \draw[thick, ->] (0,-2,0) -- (0,3.5,0) node[anchor=north west]{$y$};  % y-axis
                \draw[thick, ->] (0,0,-1) -- (0,0,2) node[anchor=south]{$z$};  % z-axis
                % origin
                \node at(-0.5,-0.3,0.05) {{\footnotesize $O$}};
                % vector u_1
                \draw[very thin, dashed, color=gray] (0.7071,0,0) -- (0.7071,0.7071,0) -- (0,0.7071,0);
                \draw[very thick, ->, color=orange] (0,0,0) -- (0.7071,0.7071,0);
                \node at(0.8,1,0) {$\bm{u}_1$};
                % vector v_2'
                \draw[very thin, dashed, color=gray] (-1,0,0) -- (-1,1,0) -- (0,1,0);
                \draw[very thin, dashed, color=gray] (-1,1,0) -- (-1,1,1);
                \draw[very thick, ->, color=green] (0,0,0) -- (-1,1,1);
                \node at(-1.2,1.2,1.1) {$\bm{v}_2'$};
                % vector v_3
                \draw[very thin, dashed, color=gray] (2,0,0) -- (2,-1,0) -- (0,-1,0);
                \draw[very thin, dashed, color=gray] (2,-1,0) -- (2,-1,1);
                \draw[very thick, ->, color=blue] (0,0,0) -- (2,-1,1);
                \node at(2.2,-1.2,1.2) {$\bm{v}_3$};
            \end{tikzpicture}
        \end{center}
        \caption{$\bm{u}_1, \bm{v}_2', \bm{v}_3$}
        \label{fig_u1v2dv3}
    \end{minipage}
\end{figure}

\begin{figure}[H]
    % left
    \begin{minipage}{0.60\textwidth}
        \begin{leftbar}
            \begin{align*}
                \bm{u}_2
                = \frac{1}{\| \bm{v}_2' \|} \bm{v}_2'
                = \frac{1}{\sqrt{3}} \begin{bmatrix} -1 \\ 1 \\ 1 \end{bmatrix} ,
            \end{align*}
        \end{leftbar}

        $\bm{v}_2'$を正規化して,$\bm{u}_2$を求めます.
        この操作は,\ref{ssec_conc}節の手順\ref{calc_uj}で$j = 2$の場合に相当します.
    \end{minipage}
    % right
    \begin{minipage}{0.39\textwidth}
        \begin{center}
            \tdplotsetmaincoords{70}{100}
            \begin{tikzpicture}[tdplot_main_coords, scale=0.85]
                % axis
                \draw[thick, ->] (-2,0,0) -- (4,0,0) node[anchor=north east]{$x$};  % x-axis
                \draw[thick, ->] (0,-2,0) -- (0,3.5,0) node[anchor=north west]{$y$};  % y-axis
                \draw[thick, ->] (0,0,-1) -- (0,0,2) node[anchor=south]{$z$};  % z-axis
                % origin
                \node at(-0.5,-0.3,0.05) {{\footnotesize $O$}};
                % vector u_1
                \draw[very thin, dashed, color=gray] (0.7071,0,0) -- (0.7071,0.7071,0) -- (0,0.7071,0);
                \draw[very thick, ->, color=orange] (0,0,0) -- (0.7071,0.7071,0);
                \node at(0.8,1,0) {$\bm{u}_1$};
                % vector u_2
                \draw[very thin, dashed, color=gray] (-0.5773,0,0) -- (-0.5773,0.5773,0) -- (0,0.5773,0);
                \draw[very thin, dashed, color=gray] (-0.5773,0.5773,0) -- (-0.5773,0.5773,0.5773);
                \draw[very thick, ->, color=teal] (0,0,0) -- (-0.5773,0.5773,0.5773);
                \node at(-0.6,0.8,0.6) {$\bm{u}_2$};
                % vector v_3
                \draw[very thin, dashed, color=gray] (2,0,0) -- (2,-1,0) -- (0,-1,0);
                \draw[very thin, dashed, color=gray] (2,-1,0) -- (2,-1,1);
                \draw[very thick, ->, color=blue] (0,0,0) -- (2,-1,1);
                \node at(2.2,-1.2,1.2) {$\bm{v}_3$};
            \end{tikzpicture}
        \end{center}
        \caption{$\bm{u}_1, \bm{u}_2, \bm{v}_3$}
        \label{fig_u1u2v3}
    \end{minipage}
\end{figure}

\begin{figure}[H]
    % left
    \begin{minipage}{0.60\textwidth}
        \begin{leftbar}
            \begin{align*}
                \bm{v}_3'
                = \bm{v}_3 - (\bm{v}_3, \bm{u}_1) \bm{u}_1 - (\bm{v}_3, \bm{u}_2) \bm{u}_2
                = \frac{5}{6} \begin{bmatrix} 1 \\ -1 \\ 2 \end{bmatrix} ,
            \end{align*}
        \end{leftbar}

        $\bm{v}_3$から$\bm{u}_1, \bm{u}_2$の成分を引いて$\bm{v}_3'$を求めます.
        これで$\bm{u}_1, \bm{u}_2, \bm{v}_3'$がそれぞれ直交するようになりました.
        この操作は,\ref{ssec_conc}節の手順\ref{calc_vjd}で$j = 3$の場合に相当します.
    \end{minipage}
    % right
    \begin{minipage}{0.39\textwidth}
        \begin{center}
            \tdplotsetmaincoords{70}{100}
            \begin{tikzpicture}[tdplot_main_coords, scale=0.85]
                % axis
                \draw[thick, ->] (-2,0,0) -- (4,0,0) node[anchor=north east]{$x$};  % x-axis
                \draw[thick, ->] (0,-2,0) -- (0,3.5,0) node[anchor=north west]{$y$};  % y-axis
                \draw[thick, ->] (0,0,-1) -- (0,0,2) node[anchor=south]{$z$};  % z-axis
                % origin
                \node at(-0.5,-0.3,0.05) {{\footnotesize $O$}};
                % vector u_1
                \draw[very thin, dashed, color=gray] (0.7071,0,0) -- (0.7071,0.7071,0) -- (0,0.7071,0);
                \draw[very thick, ->, color=orange] (0,0,0) -- (0.7071,0.7071,0);
                \node at(0.8,1,0) {$\bm{u}_1$};
                % vector u_2
                \draw[very thin, dashed, color=gray] (-0.5773,0,0) -- (-0.5773,0.5773,0) -- (0,0.5773,0);
                \draw[very thin, dashed, color=gray] (-0.5773,0.5773,0) -- (-0.5773,0.5773,0.5773);
                \draw[very thick, ->, color=teal] (0,0,0) -- (-0.5773,0.5773,0.5773);
                \node at(-0.6,0.8,0.6) {$\bm{u}_2$};
                % vector v_3'
                \draw[very thin, dashed, color=gray] (0.8333,0,0) -- (0.8333,-0.8333,0) -- (0,-0.8333,0);
                \draw[very thin, dashed, color=gray] (0.8333,-0.8333,0) -- (0.8333,-0.8333,1.6666);
                \draw[very thick, ->, color=blue] (0,0,0) -- (0.8333,-0.8333,1.6666);
                \node at(0.9,-1.1,1.8) {$\bm{v}_3'$};
            \end{tikzpicture}
        \end{center}
        \caption{$\bm{u}_1, \bm{u}_2, \bm{v}_3'$}
        \label{fig_u1u2v3d}
    \end{minipage}
\end{figure}

\begin{figure}[H]
    \begin{minipage}{0.60\textwidth}
        \begin{leftbar}
            \begin{align*}
                \bm{u}_3
                = \frac{1}{\| \bm{v}_3' \|} \bm{v}_3'
                = \frac{1}{\sqrt{6}} \begin{bmatrix} 1 \\ -1 \\ 2\end{bmatrix} .
            \end{align*}
        \end{leftbar}

        最後に$\bm{v}_3'$を正規化して$\bm{u}_3$を求めます.
        この操作は,\ref{ssec_conc}節の手順\ref{calc_uj}で$j = 3$の場合に相当します.
    \end{minipage}
    % right
    \begin{minipage}{0.39\textwidth}
        \begin{center}
            \tdplotsetmaincoords{70}{100}
            \begin{tikzpicture}[tdplot_main_coords, scale=0.85]
                % axis
                \draw[thick, ->] (-2,0,0) -- (4,0,0) node[anchor=north east]{$x$};  % x-axis
                \draw[thick, ->] (0,-2,0) -- (0,3.5,0) node[anchor=north west]{$y$};  % y-axis
                \draw[thick, ->] (0,0,-1) -- (0,0,2) node[anchor=south]{$z$};  % z-axis
                % origin
                \node at(-0.5,-0.3,0.05) {{\footnotesize $O$}};
                % vector u_1
                \draw[very thin, dashed, color=gray] (0.7071,0,0) -- (0.7071,0.7071,0) -- (0,0.7071,0);
                \draw[very thick, ->, color=orange] (0,0,0) -- (0.7071,0.7071,0);
                \node at(0.8,1,0) {$\bm{u}_1$};
                % vector u_2
                \draw[very thin, dashed, color=gray] (-0.5773,0,0) -- (-0.5773,0.5773,0) -- (0,0.5773,0);
                \draw[very thin, dashed, color=gray] (-0.5773,0.5773,0) -- (-0.5773,0.5773,0.5773);
                \draw[very thick, ->, color=teal] (0,0,0) -- (-0.5773,0.5773,0.5773);
                \node at(-0.6,0.8,0.6) {$\bm{u}_2$};
                % vector v_3'
                \draw[very thin, dashed, color=gray] (0.4082,0,0) -- (0.4082,-0.4082,0) -- (0,-0.4082,0);
                \draw[very thin, dashed, color=gray] (0.4082,-0.4082,0) -- (0.4082,-0.4082,0.8164);
                \draw[very thick, ->, color=cyan] (0,0,0) -- (0.4082,-0.4082,0.8164);
                \node at(0.5,-0.5,1.0) {$\bm{u}_3$};
            \end{tikzpicture}
        \end{center}
        \caption{$\bm{u}_1, \bm{u}_2, \bm{u}_3$}
        \label{fig_u1u2u3}
    \end{minipage}
\end{figure}

\begin{leftbar}
    \begin{align*}
        \text{(答)} \quad \left\{
        \frac{1}{\sqrt{2}} \begin{bmatrix} 1 \\ 1 \\ 0 \end{bmatrix} ,
        \frac{1}{\sqrt{3}} \begin{bmatrix} -1 \\ 1 \\ 1 \end{bmatrix} ,
        \frac{1}{\sqrt{6}} \begin{bmatrix} 1 \\ -1 \\ 2 \end{bmatrix}
        \right\} .
    \end{align*}
\end{leftbar}
以上の操作で,$\bm{R}^3$の基$\{ \bm{v}_1, \bm{v}_2, \bm{v}_3 \}$をもとに,正規直交基$\{ \bm{u}_1, \bm{u}_2, \bm{u}_3 \}$が求められました.


\begin{question}
    上の通り,教科書の解法では,$\bm{v}_1, \bm{v}_2, \bm{v}_3$の順で正規直交化を行っていました.
    これを逆順($\bm{v}_3, \bm{v}_2, \bm{v}_1$の順)に正規直交化を行ってください.
\end{question}

\begin{question}
    多項式空間$\bm{R}[x]_2$の基$\{ 1, x, x^2 \}$に対して正規直交化を行ってください.
    内積は(\ref{func_ip})式で定めたものとします.(教科書 問題6.2 \ 2.(1))

    また,得られた正規直交基が,例\ref{ex_polspaceno}の$E_0, E_1, E_2$であることを確かめてください.
\end{question}

\begin{question}
    任意の正則行列$W$が,直交行列$U$と上三角行列$R$の積で表されることを示してください.
\end{question}


%%%----------------------------------------------------------------------
\subsection{補足}
%%%----------------------------------------------------------------------
教科書の正規直交化の説明では,正規化と直交化の操作を同時に行っていましたが,正規化の操作は一番最後にまとめて行うことも可能です.
つまり,まず直交化の操作だけを行って(正規化されていない)直交基を求め,その後,求めたベクトルをそれぞれ正規化することでも正規直交基は得られます.
この方法を用いる場合には,正規化(手順\ref{calc_uj})が省略されるので,
正射影(手順\ref{calc_vjd}の$\sum$内)の形が\ref{ssec_conc}節で示したものとは変わることに注意してください.

さらに,正規化を省略する直交化の過程では,ベクトルの方向だけ気にすればよく,ベクトルのノルムは,0以外であればどんな値でも問題ないので,
求めたベクトルを自分の都合のいいように定数倍してしまっても大丈夫です
\footnote{内積の性質(2)と直交の定義より,定数はベクトルの直交に関係しないことに注意してください.}.
手計算のテストで正規直交化を行えという問題が出たときには,途中計算が楽になるので正規化は後の方がいいかもしれません(問題にもよると思いますが).

\begin{question}
    正規化を最後にまとめて行う方法での正規直交化の手順を,\ref{ssec_conc}節の手順のような形で示してください.
\end{question}

\begin{question}
    直交化の過程において,求めたベクトルを定数倍して扱ってもよい理由を説明してください.
\end{question}



%%%======================================================================
\begin{thebibliography}{9}
    \bibitem{la} 三宅敏恒「入門線形代数」培風館 ISBN978-4-563-00216-9 1991年
    \bibitem{r} 長谷川浩司「線型代数」日本評論社 ISBN978-4-535-78371-3 2004年
    \bibitem{g} 齋藤正彦「線型代数入門」東京大学出版会 ISBN978-4-13-062001-7 1966年
    \bibitem{y} 佐武一郎「線型代数学(新装版)」裳華房 ISBN978-4-7853-1316-6 2015年
\end{thebibliography}

\end{document}

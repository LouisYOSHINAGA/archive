%%% 20.5 %%%

\begin{frame}
    \frametitle{20.5 Supervised Learning of Word Sentiment}
    \begin{itemize}
        \item Online review(スコア,テキストの組)からlexiconを構成
        \begin{itemize}
            \item 高評価のレビューほどpositiveなwordが現れやすい
            \item 多段評価なので分布が得られる
            \yvspace{-0.25}
            \yfig{width=0.8\textwidth}{figure/05/fig_20_9.png}
            \item word $w$と評価$c$の関連度を計算
        \end{itemize}
    \end{itemize}
    %
    \begin{columns}
        \begin{column}[T]{0.05\textwidth}
            \relax
        \end{column}
        %
        \begin{column}[T]{0.45\textwidth}
            \yvspace{-0.5}
            \footnotesize
            \begin{align*}
                p(w \mid c) &= \frac{count(w, c)}{\sum_{w \in c} count(w, c)} \\
                &\ygraytxt{= \frac{\mathrm{Count}(c)}{\mathrm{Total}(c)} = \mathrm{Pr}_{\mathrm{IMDB}}(w \mid c)}
            \end{align*}
            \yvspace{-0.75}
            \begin{yalign*}
                PottsScore(w) &= \frac{p(x \mid c)}{\sum_{c} p(w \mid c)} \\
                &\ygraytxt{= \mathrm{Pr}_{\mathrm{IMDB}}(c \mid w)}
            \end{yalign*}
        \end{column}
        %
        \begin{column}[T]{0.45\textwidth}
            \yvspace{-1.5}
            \yfigcap{width=\textwidth}{figure/05/potts_table.png}{\scriptsize $w=\textrm{`disappointing'}$の場合 \ycite{Potts, 2011}}
        \end{column}
    \end{columns}
\end{frame}


\begin{frame}
    \frametitle{20.5 Supervised Learning of Word Sentiment}
    \yvspace{-0.5}
    \begin{itemize}
        \item word$w$と評価$c$の関連度をvisualize(Potts diagram)
        \begin{itemize}
            \item 曲線の形をaffective meaningのタイプと見做す
        \end{itemize}
    \end{itemize}
    \yvspace{-1}
    %
    \begin{columns}
        \begin{column}[T]{0.5\textwidth}
            \yfig{width=\textwidth}{figure/05/fig_20_10.png}
            \yvspace{-0.5}
            \begin{itemize}
                \footnotesize
                \item strongly positive/negative:J-shape
                \item weakly positive/negative:hump-shape
            \end{itemize}
        \end{column}
        %
        \begin{column}[T]{0.5\textwidth}
            \yfig{width=0.975\textwidth}{figure/05/fig_20_11.png}
            \yvspace{-0.5}
            \begin{itemize}
                \yinner{0.5}
                \footnotesize
                \item empahtics:J/U-shape
                \item attenuators:hump-shape
            \end{itemize}
        \end{column}
    \end{columns}
\end{frame}


\begin{frame}
    \frametitle{20.5 Supervised Learning of Word Sentiment}
    \yhead{20.5.1 Log Odds Ration Informative Dirichlet Prior}
    \yvspace{0.25}
    %
    \begin{itemize}
        \yinner{1}
        \item 1つのカテゴリでのみ使用される 特徴的なwordの検出
            \yvspace{0.25}
        \item word $w$は コーパス$i, j$のどちらに より現れるか
        \begin{itemize}
            \yinner{1}
            \item 対数オッズ比 {\footnotesize(\ $\mathrm{odds}= p/(1-p)$\ )}:
            \begin{yalign*}
                \mathrm{lor}(w) &= \log\left(\frac{p^{i}(w)}{1-p^{i}(w)}\right) - \log\left(\frac{p^{j}(w)}{1-p^{j}(w)}\right) \\
                &= \log\left(\frac{f^{i}_{w}}{n^i-f^{i}_{w}}\right) - \log\left(\frac{f^{j}_{w}}{n^j-f^{j}_{w}}\right)
            \end{yalign*}
            where $n^i$:コーパス$i$の総word数 \\
            \yhspace{2.25} $f^{i}_{w}$:コーパス$i$内のword $w$の数 \\
            \yvspace{0.5}
            %
            \item 高頻出・低頻出単語への対応 \\
            $\to$ log odds ratio informative Dirichlet prior \ycite{Monroe, 2008}
        \end{itemize}
    \end{itemize}
    %
\end{frame}


\begin{frame}
    \frametitle{20.5 Supervised Learning of Word Sentiment}
    \begin{itemize}
        \item large background corpusを考慮:
        {\small
            \begin{align*}
                \hat{\delta}^{(i-j)}_{w} =
                    \log\left(\frac{f^{i}_{w} + \yorangetxt{\alpha_{w}}}{n^i + \yorangetxt{\alpha_{0}}-(f^{i}_{w} + \yorangetxt{\alpha_{w}})}\right)
                    - \log\left(\frac{f^{j}_{w} + \yorangetxt{\alpha_{w}}}{n^j + \yorangetxt{\alpha_{0}}-(f^{j}_{w} + \yorangetxt{\alpha_{w}})}\right)
            \end{align*}
        }
        {\footnotesize
            where $\alpha_0$:background corpusの総word数 \\
                \yhspace{2.8} $\alpha_w$:background corpus内のword $w$の数
        }
    \end{itemize}
    %
    \begin{itemize}
        \item 分散を近似:
        {\small
            \begin{align*}
                \sigma^{2}\left(\hat{\delta}^{(i-j)}_{w}\right) \approx \frac{1}{f^{i}_{w} + \alpha_{w}} + \frac{1}{f^{j}_{w} + \alpha_{w}}
            \end{align*}
        }
    \end{itemize}
    %
    \begin{itemize}
        \item z変換:
        {\small
            \begin{yalign*}
                \frac{\hat{\delta}^{(i-j)}_{w}}{\sqrt{\sigma^{2}\left(\hat{\delta}^{(i-j)}_{w}\right)}}
            \end{yalign*}
        }
    \end{itemize}
\end{frame}


\begin{frame}
    \frametitle{20.5 Supervised Learning of Word Sentiment}
    \yfigcap{width=0.95\textwidth}{figure/05/fig_20_12.png}{$\bigstar$1/$\bigstar$5のレビューと関連するword \ycite{Jurafsky+, 2014}}
\end{frame}

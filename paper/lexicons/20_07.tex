%%% 20.7 %%%

\begin{frame}
    \frametitle{20.7 Using Lexicons for Affect Recognition}
    \begin{itemize}
        \item 大規模データセット
        \begin{itemize}
            \item 何かしらの修正が必要
        \end{itemize}
    \end{itemize}
    %
    \begin{itemize}
        \item 例:言語情報と個性・性別・年齢の関連 \ycite{Schwartz+, 2013}
        \begin{itemize}
            \item Facebookの投稿{\footnotesize(700million words)}をn-gram($n=1,2,3$)の形で使用
                \yvspace{0.2}
            \item 1\%以上のユーザに使用されているword・phraseのみ使用
                \yvspace{0.2}
            \item 2-gram, 3-gram:PMIが充分大きいphraseのみ使用
                \begin{yalign*}
                    \mathrm{pmi}(phrase) = \log \frac{p(phrase)}{\displaystyle\prod_{w \in phrase} p(w)}
                \end{yalign*}
            \item 尤度を計算:
                \begin{yalign*}
                    \yhspace{4}
                    p(phrase \mid subject) = \frac{\mathrm{freq}(phrase, subject)}{\displaystyle\sum_{phrase' \in \mathrm{vocab}(subject)} \mathrm{freq}(phrase', subject)}
                \end{yalign*}
        \end{itemize}
    \end{itemize}
\end{frame}


\begin{frame}
    \frametitle{20.7 Using Lexicons for Affect Recognition}
    \begin{itemize}
        \item training dataがsparser,test setに似ていない
        \begin{itemize}
            \item lexicon $L$を利用してdocument $x$の特徴量$f_L$を作成
            %
            \begin{columns}
                \begin{column}[T]{0.05\textwidth}
                    \relax
                \end{column}
                %
                \begin{column}[T]{0.80\textwidth}
                    \yfig{width=0.8\textwidth}{figure/07/feature.png}
                    \yvspace{0.5}
                \end{column}
                %
                \begin{column}[T]{0.05\textwidth}
                    \relax
                \end{column}
            \end{columns}
            \yvspace{0.25}
            %
            \item 特性関数の条件として使用:
                {\small
                    \begin{yalign*}
                        \yhspace{2}
                        f_{L}(c, x) = \begin{cases}
                                1 & \mathrm{if} \, \exists w, w \in L \ \&\ w \in x \ \&\ class = c \\
                                0 & \mathrm{otherwise}
                        \end{cases}
                    \end{yalign*}
                }
            %
            \item 数え上げの条件として使用:
                {\small
                    \begin{yalign*}
                        f_{L} = \sum_{w \in L} \theta_{w}^{L} \ count(w)
                    \end{yalign*}
                }
        \end{itemize}
    \end{itemize}
\end{frame}

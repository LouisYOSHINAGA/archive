%%% introduction %%%

\begin{frame}
    \frametitle{Introduction}
    \begin{itemize}
        \item 主題:\yorangetxt{Affective meaning}
        \begin{itemize}
            \item affective:emotion, sentiment, personality, mood, atitudes
            \item affective statesの分類 \ycite{Scherer, 2000}
            \yfig{width=0.8\textwidth}{figure/00/fig_20_1.png}
        \end{itemize}
        %
        % \item 目的:affective statesの抽出器を設計
        % \begin{itemize}
        %     \item Chapter 4:Sentiment Analysis $\to$ Attitude
        % \end{itemize}
    \end{itemize}
\end{frame}


\begin{frame}
    \frametitle{Introduction}
    \begin{itemize}
        \item affective statesの有用性
        \begin{itemize}
            \yinner{-0.5}
            \item emotion, mood:学習支援システム,ヘルプライン,twitter・blog,小説
            \item interpersonal stance:会議要約(hot spot検出)
            \item personality:ユーザの性格検出
            \item attect生成:対話エージェントの感情付与
        \end{itemize}
    \end{itemize}
    \yvspace{1.25}
    %
    \yhead{用語}
    \begin{itemize}
        \yinner{1}
        \item \yorangetxt{affective lexicons} / \yorangetxt{sentiment lexicons}
        \begin{itemize}
            \yinner{1}
            \item affectやsentimentへの手がかりとなる 特定のwordのリスト
        \end{itemize}
        %
        \item connotations
        \begin{itemize}
            \yinner{1}
            \item 書き手・読み手のemotion, sentiment, opinions, evaluationsに\\
                関連するもの
        \end{itemize}
    \end{itemize}
\end{frame}

\documentclass[dvipdfmx]{beamer}

\usepackage{bxdpx-beamer}
\usepackage{pxjahyper}
\usepackage{minijs}
\usepackage[T1]{fontenc}
\usepackage{lmodern}
\usepackage{xcolor}
\usepackage{amsmath, amssymb, mathtools, bm}

\DeclareFontShape{JY1}{gt}{m}{it}{<->ssub*gt/m/n}{}
\DeclareFontShape{JT1}{gt}{m}{it}{<->ssub*gt/m/n}{}

\renewcommand{\kanjifamilydefault}{\gtdefault}
\newcommand{\fp}[2]{\frac{\partial #1}{\partial #2}}
\newcommand{\tp}{{}^t\!}
\renewcommand{\b}[2]{b_{#1}^{(#2)}}
\renewcommand{\d}[2]{\delta_{#1}^{(#2)}}
\renewcommand{\u}[2]{u_{#1}^{(#2)}}
\newcommand{\w}[2]{w_{#1}^{(#2)}}
\newcommand{\z}[2]{z_{#1}^{(#2)}}
\newcommand{\bd}[2]{\bm{\delta}_{#1}^{(#2)}}
\newcommand{\bu}[2]{\bm{u}_{#1}^{(#2)}}
\newcommand{\bz}[2]{\bm{z}_{#1}^{(#2)}}
\newcommand{\bup}[2]{\bm{#1}^{(#2)}}
\newtheorem{thm}{定理}

\usefonttheme{professionalfonts}

\usetheme{Madrid}
\setbeamertemplate{navigation symbols}{}
\setbeamertemplate{itemize item}[circle]
\setbeamertemplate{itemize subitem}[circle]
\setbeamertemplate{enumerate item}[default]


\title{誤差逆伝播法}
\author{吉永 塁}
\date{2020年5月15日}


\begin{document}

\begin{frame}[plain,noframenumbering]
    \titlepage
\end{frame}


\begin{frame}
    \frametitle{記号設定}
    \begin{itemize}
        \item $L$層のニューラルネットワークを考える.
        \item 第$l$層のユニット数を$\upsilon(l)$とおく.
        \item 各ユニットの入出力について,
            \begin{align*}
                \z{j}{l}
                = f(\u{j}{l})
                = f\left( \sum_{i=0}^{\upsilon(l-1)} \w{ji}{l} \z{i}{l-1} \right)
            \end{align*}
            \begin{itemize}
                \item $z$は入力,$u$は出力を表す.
                \item 右肩括弧内の数字は第何層かを,下添字は何番目のユニットかを示す.
                \item $w$は重み.
                \item バイアスは$\w{j0}{l} = \b{j}{l}$,$\z{0}{l} \equiv 1$として重みの一部とする.
                \item $f$は活性化関数.
            \end{itemize}
        \item $\bm{w}$ : 全てのパラメータ(重み,バイアス)を成分に持つベクトル.
        \item $\bm{1}_{n} \coloneqq \sum_{i=1}^{n} \bm{e}_i$ : 全成分が1の$n$次元ベクトル
    \end{itemize}
\end{frame}


\begin{frame}
    \frametitle{連鎖律}
    \begin{thm}[連鎖律]
        $U, W$がそれぞれ$\mathbb{R}^n, \mathbb{R}^m$の開集合であり,
        二つの関数$f: U \to \mathbb{R}^n$,$g: W \to \mathbb{R}^p$が合成可能,
        つまり$f(U) \subset W$であるとする.

        いま,$f$が$\bm{x} \in U$で微分可能で,$g$が$\bm{y} = f(\bm{x})$で微分可能であるとすれば,
        合成関数$\varphi = g \circ f$は$\bm{x}$で微分可能で,
        \begin{align*}
            &\fp{\varphi_r}{x_j}(\bm{x}) = \sum_{i=1}^{m} \fp{g_r}{y_i}(\bm{y}) \fp{f_i}{x_j}(\bm{x}) \\
            &(1 \leq r \leq p, 1 \leq j \leq n)
        \end{align*}
        但し,添字$i$は第$i$成分を表す.
    \end{thm}
    証明略.
\end{frame}


\begin{frame}
    \frametitle{誤差逆伝播法}
    \begin{itemize}
        \item 勾配降下法では重み更新において勾配$\nabla E$が必要.
        ここで誤差関数$E(\bm{w})$について,
        \begin{align*}
            \begin{dcases*}
                E(\bm{w}) = \sum_{n=1}^{N} E_n(\bm{w}) & (バッチ学習) \\
                E(\bm{w}) = \frac{1}{|D_t|} \sum_{n \in D_t} E_n(\bm{w}) & (ミニバッチ学習)
            \end{dcases*}
        \end{align*}
        より,何れも$\nabla E_n(\bm{w})$を計算できれば良い.
        \item 勾配計算はそのまま行うと面倒.
        例えば,二乗誤差の場合,
            \begin{align*}
                \fp{E_n}{\w{ji}{l}} = \tp(\bm{y}(\bm{x}_n) - \bm{d}_n) \fp{\bm{y}}{\w{ji}{l}}
            \end{align*}
            となり,$\bm{y}$について,活性化関数のネストが深くなるので,連鎖律を繰り返し適用しなければいけない.
    \end{itemize}
\end{frame}


\begin{frame}
    \frametitle{誤差逆伝播法}
    第$l$層の重み$\w{ji}{l}$について,
    \begin{align}
        \fp{E_n}{\w{ji}{l}} = \fp{E_n}{\u{j}{l}} \fp{\u{j}{l}}{\w{ji}{l}} \label{e4_9}
    \end{align}
    (\ref{e4_9})式右辺第一項について,
    \begin{align}
        \fp{E_n}{\u{j}{l}} = \sum_{k=0}^{\upsilon(l+1)} \fp{E_n}{\u{k}{l+1}} \fp{\u{k}{l+1}}{\u{j}{l}}  \label{e4_10}
    \end{align}
    ここで,記号
    \begin{align}
        \d{j}{l} \coloneqq \fp{E_n}{\u{j}{l}}  \label{e4_11}
    \end{align}
    を導入する.$\delta$は誤差と呼ばれる.
\end{frame}


\begin{frame}
    \frametitle{誤差逆伝播法}
    (\ref{e4_10})式右辺第二項について,$\u{k}{l+1} = \sum_{i} \w{ki}{l+1} f(\u{i}{l})$に注意して,
    \begin{align*}
        \fp{\u{k}{l+1}}{\u{j}{l}} = \w{kj}{l+1} f'(\u{j}{l})
    \end{align*}
    よって式(\ref{e4_10})は,
    \begin{align}
        \d{j}{l} = f'(\u{j}{l}) \sum_{k=0}^{\upsilon(l+1)} \d{k}{l+1} \w{kj}{l+1}  \label{e4_12}
    \end{align}
    と表される.
    この式は,$\d{j}{l}$が$\d{k}{l+1} (k = 0, \ldots , \upsilon(l+1))$により計算できることを意味する.

    \begin{itemize}
        \item 微分計算は出力層から入力層へと順に計算できる.
        \item 順伝播とは逆向きに伝播するので誤差逆伝播と呼ばれる.
    \end{itemize}
\end{frame}


\begin{frame}
    \frametitle{誤差逆伝播法}
    また,逆伝播の最初の値
    \begin{align*}
        \d{j}{L} = \fp{E_n}{\u{j}{L}}
    \end{align*}
    は陽に計算される.

    (\ref{e4_9})式右辺第二項について,
    $\u{j}{l} = \sum_{k} \w{jk}{l} \z{k}{l-1}$に注意して,
    \begin{align*}
        \fp{\u{j}{l}}{\w{ji}{l}} = \z{i}{l-1}
    \end{align*}
    以上より,(\ref{e4_9})式は,
    \begin{align}
        \fp{E_n}{\w{ji}{l}} = \d{j}{l} \z{i}{l-1}  \label{e4_13}
    \end{align}
    と表される.
\end{frame}


\begin{frame}
    \frametitle{誤差逆伝播法}
    以上より,訓練サンプル$(\bm{x}_n, \bm{d}_n)$が与えられたとき,この訓練サンプルについての誤差関数$E_n$の勾配は次の手順で得られる.
    \begin{block}{誤差逆伝播法による勾配計算}
        \begin{enumerate}
            \item $\bm{z}^{(1)} = \bm{x}_n$として各層の入出力$\bm{u}^{(l)}, \bm{z}^{(l)} (l = 2,\ldots, L)$を計算.\\
            このとき計算は入力層から順に行われる.[順伝播]
            \item 出力層(第$L$層)の誤差$\d{j}{L}$を計算.
            \item 中間層について,(\ref{e4_12})式より誤差$\d{j}{l} (l = 2, \ldots , L-1)$を計算.\\
            このとき計算は出力層から順に行われる.[逆伝播]
            \item (\ref{e4_13})式より各層の重み$\w{ji}{l}$に関する微分を計算.
        \end{enumerate}
    \end{block}

    複数個の訓練サンプルについての誤差の総和は
    \begin{align*}
        \fp{E}{\w{ji}{l}} = \sum_{n} \fp{E_n}{\w{ji}{l}}
    \end{align*}
    より,単に加算すればよい.
\end{frame}


\begin{frame}
    \frametitle{誤差逆伝播法}
    計算量について考える.
    ネットワークの重みの総数を$W$とおく.

    このとき,順伝播+逆伝播により勾配$\nabla E$を求めるのには,$\mathcal{O}(W)$の計算が必要となる.
    \begin{quote}
    $\because$ 一般的に,$(\text{重み数}) >> (\text{ユニット数})$であることに注意する.
    順伝播計算の大部分は,$\u{j}{l} = \sum_{i} \w{ji}{l} \z{i}{l-1}$であるので,計算量は$W$による.

    逆伝播計算についても,$\d{j}{l} = f'(\u{j}{l}) \sum_{k} \d{k}{l+1} \w{kj}{l+1}$より同様.
    \end{quote}
    このことより,数値微分
    \begin{align*}
        \fp{E_n}{\w{ji}{l}} = \frac{1}{\varepsilon} (E_n(\w{ji}{l} + \varepsilon) - E_n(\w{ji}{l})) + O(\varepsilon)
    \end{align*}
    は,全ての重みについて$E_n(\w{ji}{l} + \varepsilon)$を計算する必要があるため,全体では$\mathcal{O}(W^2)$となる.

    数値微分は計算効率は良くないが,逆伝播の実装の確認として有用.
\end{frame}


\begin{frame}
    \frametitle{誤差逆伝播法}
    ここまでの計算を行列形式で表す.
    但し,バイアスと重みを区別して扱う.
    \begin{itemize}
        \item $\bm{X} = \begin{bmatrix} \bm{x}_1 \cdots \bm{x}_N\end{bmatrix}$
        : $\bm{x}_n$は$n$番目の訓練サンプル.
        \item $\bup{U}{l} = \begin{bmatrix} \bu{1}{l} \cdots \bu{N}{l} \end{bmatrix}$
        : $\bu{n}{l}$は$\bm{x}_n$についての$\tp \begin{bmatrix}\u{1}{l} \cdots \u{\upsilon(l)}{l}\end{bmatrix}$
        \item $\bup{Z}{l} = \begin{bmatrix} \bz{1}{l} \cdots \bz{N}{l} \end{bmatrix}$
        : $\bz{n}{l}$は$\bu{n}{l}$と同様.
        \item $\bup{W}{l} = \begin{bmatrix} \w{ji}{l} \end{bmatrix}_{\substack{1 \leq j \leq \upsilon(l) \\ 1 \leq i \leq \upsilon(l-1)}}$
        : ユニット間に接続がなければ0.
        \item $\bup{b}{l} = \tp \begin{bmatrix} b_1 \cdots b_{\upsilon(l)} \end{bmatrix}$
    \end{itemize}
    以上より,順伝播計算は$\bup{U}{1} = \bm{X}$として,
    \begin{align}
        \bup{U}{l} &= \bup{W}{l} \bup{Z}{l-1} + \bup{b}{l} \, \tp\bm{1}_{N} \\
        \bup{Z}{l} &= f^{(l)}\left( \bup{U}{l} \right)
    \end{align}
    と表される$(l = 2, \ldots , L)$.但し,活性化関数$f^{(l)}$は各成分に適応されることとする.
\end{frame}


\begin{frame}
    \frametitle{誤差逆伝播法}
    逆伝播について,
    \begin{itemize}
        \item $\bm{Y} = \begin{bmatrix} \bm{y}_1 \cdots \bm{y}_N \end{bmatrix}$
        : $\bm{y}_n$は$\bm{x}_n$についての出力.
        \item $\bm{D} = \begin{bmatrix} \bm{d}_1 \cdots \bm{d}_N \end{bmatrix}$
        : $\bm{d}_n$は$\bm{x}_n$についての出力目標.
        \item $\bup{\Delta}{l} = \begin{bmatrix} \bd{1}{l} \cdots \bd{N}{l} \end{bmatrix}$
        : $\bd{n}{l}$は$\bm{x}_n$についての$\tp \begin{bmatrix}\d{1}{l} \cdots \d{\upsilon(l)}{l}\end{bmatrix}$
    \end{itemize}
    とおく.
    以上より,逆伝播計算は$\bup{\Delta}{L} = \bm{D} - \bm{Y}$として,
    \begin{align}
        \bup{\Delta}{l} = f^{(l)'}(\bup{U}{l}) \odot (\tp\bup{W}{l+1} \bup{\Delta}{l+1})  \label{e4_15}
    \end{align}
    と表される$(l = L-1, \ldots , 2)$.但し,$\odot$はアダマール積.
\end{frame}


\begin{frame}
    \frametitle{誤差逆伝播法}
    誤差関数の勾配計算について,
    \begin{itemize}
        \item $\partial\bup{W}{l} = \begin{bmatrix} \fp{E}{\w{ji}{l}} \end{bmatrix}_{\substack{1 \leq j \leq \upsilon(l) \\ 1 \leq i \leq \upsilon(l-1)}}$,
        $\partial\bup{b}{l} = \tp \begin{bmatrix} \fp{E}{\b{1}{l}} \cdots \fp{E}{\b{\upsilon(l)}{l}} \end{bmatrix}$
    \end{itemize}
    とおく.
    以上より,
    \begin{align*}
        \partial\bup{W}{l} &= \frac{1}{N} \bup{\Delta}{l} \, \tp\bup{Z}{l-1} \\
        \partial\bup{b}{l} &= \frac{1}{N} \bup{\Delta}{l} \bm{1}_N
    \end{align*}
    となり,パラメータ更新は,
    \begin{align}
        \bup{W}{l} &\leftarrow \bup{W}{l} - \varepsilon \, \partial\bup{W}{l}  \label{rw}\\
        \bup{b}{l} &\leftarrow \bup{b}{l} - \varepsilon \, \partial\bup{b}{l}  \label{rb}
    \end{align}
    と表される.
    重みの更新について,重み減衰やモメンタムを用いる場合には,(\ref{rw})(\ref{rb})式右辺第二項を適当に修正する.
\end{frame}


\end{document}
